\title{QuantEx: Use-case methods}
\author{
        QuantEx Team
}
\date{\today}

\documentclass[11pt, a4paper]{article}
\usepackage{amsmath}
\usepackage[usenames, dvipsnames]{xcolor}
\usepackage{listings}
\usepackage{url}
\usepackage{verbatim}
\usepackage{dsfont}
\usepackage[braket, qm]{qcircuit}
\usepackage[paperwidth=210mm,paperheight=297mm,centering,hmargin=2cm,vmargin=2.5cm]{geometry}
\renewcommand{\baselinestretch}{1.1} 
%\usepackage{tikz}
%\usetikzlibrary{quantikz}

\begin{document}
\maketitle

\begin{abstract}
The Quantum Fourier Transform (QFT) is used as a building block in many quantum algorithms and for validation and benchmarking of circuit simulators. For these reasons it makes sense to use it as a use case for QuantEx.
\end{abstract}


\section{QFT use-case}
The Quantum Fourier Transform (QFT) is used as a building block in many quantum algorithms and for validation and benchmarking of circuit simulators. 
For these reasons it makes sense to use it as a use case for QuantEx.

\subsection{Introduction}
The QFT is the quantum analogue of the inverse Fourier transform. Given an input state $| \psi \rangle = \sum_{j=0}^{2^n-1} c_j | j \rangle $ defined over $n$ qubits, the 
QFT operator F acts on $\psi\rangle$ as

 $$F | \psi \rangle =  | \phi \rangle = \sum_{j=0}^{2^n-1} d_j | j \rangle$$ 
 
 where 
 
 $$d_j = \frac{1}{\sqrt{2^n}}\sum_{k=0}^{2^n-1} c_k e^{\frac{2\pi j}{2^n}jk}$$


\subsection{Quantum circuit description and construction}
For the purposes of this use case we start with the quantum circuit defined in \cite{coppersmith_approximate_2002} to calculate the QFT of a given input state. For this implementation
 the number of gates required is quadratic in the number of qubits. While not as efficient as other known circuits \cite{hales_improved_2000,kitaev_quantum_1995}, it is a very simple 
 circuit and should serve as a starting point.
 
 The circuit consists of Hadamard gates, controlled rotation gates and swap gates. For 3 qubits, the circuit looks like
 
\begin{equation*}
    \Qcircuit @C=1.0em @R=0.0em @!R {
                \lstick{ q_{0} : \ket{0} } & \gate{H} & \control \qw & \dstick{\frac{\pi}{4}}\qw & \control \qw & \dstick{\frac{\pi}{8}}\qw \barrier[0em]{2} & \qw & \qw & \qw & \qw \barrier[0em]{2} & \qw & \qw \barrier[0em]{2} & \qw & \qswap & \qw & \qw\\
                \lstick{ q_{1} : \ket{0} } & \qw & \ctrl{-1} & \qw & \qw & \qw & \qw & \gate{H} & \control \qw & \dstick{\frac{\pi}{4}}\qw & \qw & \qw & \qw & \qw & \qw & \qw\\
                \lstick{ q_{2} : \ket{0} } & \qw & \qw & \qw & \ctrl{-2} & \qw & \qw & \qw & \ctrl{-1} & \qw & \qw & \gate{H} & \qw & \qswap \qwx[-2] & \qw & \qw\\
         }
\end{equation*}

Larger circuits will have the same structure just with more iterations.
 
\subsection{Testing procedure}
The aim of using this circuit is twofold:

\begin{enumerate}
\item It provides validation of the correctness of the simulation results
\item It can be used to benchmark simulator performance and compare to other simulator platforms
\end{enumerate}

\subsubsection{Validation}
Simulating the QFT with the default input state which initialises all qubits to state $|0\rangle$ produces an equal superposition over 
all qubit configurations. This is not very effective for validating the correctness of the underlying simulation platform. To better test
the simulator we prepare the initial state in a less trivial state through the use of a simple parameterised ansatz circuit. 

%This simple 
%ansatz circuit we choose consists of a layer of single qubit unitary gates with randomly selected parameters followed by a layer of
%entangling gates between neighbouring qubits. While not capable of producing highly entangled states it should suffice for these purposes.

Validation then proceeds by measuring the amplitudes of a set of output configurations and comparing these to their expected amplitudes. For smaller circuits 
it will be possible to do this for every possibly configuration, but for larger circuits it will be necessary to use a sample of these.

\subsubsection{Benchmarking}
For benchmarking, we wish to compute the time taken for running the circuit with a range of circuit sizes and on different numbers of CPUs/GPUs and nodes. From 
these, the scaling behaviour can be characterised. It is important to be aware that if using compression techniques (to reduce the dimension of intermediate tensors) 
the runtime of the simulation will depend not only on the number of qubits and resources used, but also on the form and parameters of the ansatz circuit.

\subsubsection{Tasks}
Sketch of the tasks envisaged to implement this use cases are as follows:

\begin{enumerate}
\item Script to generate QASM for QFT circuit with a given number of qubits
\item Script to generate QASM for ansatz circuit of given size and parameters (or to allow simulator start from arbitrary input state)
\item Script to run simulator and calculate output amplitudes for a set of configurations
\item Script to compare amplitudes and estimate the accuracy of the simulation
\item Ability to accurately record run time
\item Collate timings and accuracies, plot results and document conclusions
\end{enumerate}

\section{NCU use-case}
The application of two-qubit controlled unitary gates (operators) is essential to exploit the full extent of quantum devices. The CNOT (controlled $\sigma_x$) gate is a prime example, allowing the application of $\sigma_x$ to a target qubit if the controlling qubit is $\vert 1 \rangle$. A generalised variant of this gate type extends to $n$-controlling qubits, applying a unitary gate $U$ if they are all $\vert 1 \rangle$. This NCU ($n$-controlled unitary) gate structure can be an important tool in developing many algorithms. As such, we implement methods to generate circuits for realising such NCU gates.

\subsection{Introduction}
A fundamental element of computer science is branching, where the flow of control in a given program can take different paths based on some controlling logic. The \verb+if-else+ statements can be use to implement such logical control in classical devices. In quantum computing environments, the closest analogous operation are those of controlled unitary ($CU$) gates. The purpose of a CU gate is to apply a given unitary gate to a target qubit depending upon the values of another controlling qubit. The smallest variant of these such gates are the 2-qubit controlled gates, with the controlled-NOT (CNOT), controlled-Z (CZ), and controlled Y-rotation (CR$_{y}$) being three such examples. 

It is known that all larger CU gates with $n$-control qubits (NCU) can be decomposed into fundamental 2-qubit controls~\cite{barenco_elementary_1995}. These gates form the back-bone of many other algorithms, with the diffusion gate in Grover's search algorithm~\cite{grover_1996} being a notable example. As such, having a scalable implementation allowing the use of these gates is important for evaluating many real-world examples and problems.

As a consequence of implementing this gate structure we can gain the following outcomes:

\begin{enumerate}
\item It enables the implementation of generalised variants of additional quantum algorithms.
\item It provides functionality unavailable in many other quantum software suites.
\item It allows for an easy means to generate deep quantum circuits, useful when verifying different tensor contraction and approximation strategies.
\item Examines the ability of the underlying suite to handle non-local 2-qubit gates at scale.
\end{enumerate}


\subsection{Operations}
As an example, we can take the 2 qubit controlled-NOT ($\textrm{CNOT}_{c,t}$) gate, wherein the application of the NOT ($\sigma_x$) gate to a target qubit, $t$, is controlled by the values of another qubit, $c$, defined as:

\begin{align}
\textrm{CNOT}_{c,t}: \vert c \rangle \vert t \rangle 
\rightarrow 
(\vert 0 \rangle\langle 0 \vert \otimes \mathds{1} + \vert 1 \rangle\langle 1 \vert \otimes \sigma_x)\vert c \rangle \otimes \vert t.\rangle\end{align}

Applying this operation to all two qubit patterns yields the following truth table:
\begin{align}
\textrm{CNOT}_{0,1} \vert 00 \rangle & \rightarrow \vert 00 \rangle \\
\textrm{CNOT}_{0,1}\vert 01 \rangle & \rightarrow \vert 01 \rangle \\
\textrm{CNOT}_{0,1}\vert 10 \rangle & \rightarrow \vert 11 \rangle \\
\textrm{CNOT}_{0,1}\vert 11 \rangle & \rightarrow \vert 10 \rangle.
\end{align}


To extend the above CNOT gate to an $n$-controlled version, we require it to perform the following operation:
\begin{align}
\textrm{nCX}_{c_{1},\dots c_{n} ,t} : &\vert c_1 \rangle \dots\vert c_{n} \rangle \vert t \rangle \rightarrow \\ 
&\vert c_1 \rangle \dots \vert c_{n} \rangle \vert t \oplus (c_{1} \wedge \dots \wedge c_{n}) \rangle .
\end{align}


\subsection{Circuit description and construction}

As a circuit, the above CNOT gate is represented as

\begin{equation*}
\Qcircuit @C=1em @R=1em {
    & \ctrl{1} & \qw \\
    & \targ &  \qw
},
\end{equation*}
which is further expressed as an $n$-controlled variant by
\begin{equation*}
\Qcircuit @C=1em @R=1em {
    & \ctrl{1} & \qw \\
    & \ctrl{1} & \qw \\
    & \ctrl{1} & \qw \\
    & \vdots\qw & \qw \\
    & \ctrl{1} & \qw \\
    & \targ &  \qw
}
\end{equation*}
The above circuit can be arbitrarily extended to $n$ controlling qubits for an $n+1$ circuit.

To realise the $n$-controlled variant, we must first decompose the circuit into $2$-qubit gates. As a generalisation to any arbitrary number of control lines and unitary gate, the following decomposition can be applied iteratively to decompose any $n$-controlled unitary to 2-qubit gates~\cite{barenco_elementary_1995}:
\begin{align}\label{eqn:default_decomp}
\Qcircuit @C=.5em @R=0em @!R {
& \ctrl{1} & \qw & & & \qw & \ctrl{1} & \qw & \ctrl{1} & \ctrl{2} & \qw\\
& \ctrl{1} & \qw & & & \qw & \ctrl{1} & \qw & \ctrl{1} & \ctrl{2} & \qw\\
& \ctrl{1} & \qw & \push{\rule{.3em}{0em}=\rule{.3em}{0em}} & & \qw & \ctrl{1} & \qw & \ctrl{1} & \ctrl{2} & \qw\\
& \ctrl{1} & \qw & & & \qw & \ctrl{1} & \qw & \ctrl{1} & \ctrl{2} & \qw\\
& \ctrl{1} & \qw & & & \ctrl{1} & \targ & \ctrl{1} & \targ & \qw & \qw\\
& \gate{U} & \qw & & & \gate{\sqrt[\leftroot{-2}\uproot{2}]{U}} & \qw & \gate{\sqrt[\leftroot{-2}\uproot{2}]{U}^\dag} & \qw & \gate{\sqrt[\leftroot{-2}\uproot{2}]{U}} & \qw
},
\end{align}
This decomposition procedure generates a quadratic number of gate calls. For certain NCU gates, more efficient decompositions can be used. For a $3CU$ gate, the following decomposition reduces the call depth from 17 to 13 gates:
\newcommand{\rtU}[1][U]{\sqrt[4]{#1}}
\newcommand{\rtUd}[1][U]{\sqrt[4]{#1}^\dagger}
\begin{align}\label{eqn:3cu_decomp}
\Qcircuit @C=.5em @R=0em @!R {
& \ctrl{1} & \qw & & & \qw & \ctrl{3} & \ctrl{1} & \qw & \ctrl{1} & \qw & \qw & \qw & \ctrl{2} & \qw & \qw & \qw & \ctrl{2} & \qw & \qw \\
& \ctrl{1} & \qw & & & \qw & \qw & \targ & \ctrl{2} & \targ & \ctrl{2} & \ctrl{1} & \qw & \qw & \qw & \ctrl{1} & \qw & \qw & \qw & \qw \\
& \ctrl{1} & \qw & \push{\rule{.3em}{0em}=\rule{.3em}{0em}}
 & & \qw & \qw & \qw & \qw & \qw & \qw & \targ & \ctrl{1} & \targ & \ctrl{1} & \targ & \ctrl{1} & \targ & \ctrl{1} & \qw \\
& \gate{U} & \qw & & &\qw & \gate{\rtU{}} & \qw & \gate{\rtUd{}} & \qw & \gate{\rtU{}} & \qw & \gate{\rtUd{}} & \qw & \gate{\rtU{}} & \qw & \gate{\rtUd{}} & \qw & \gate{\rtU{}} & \qw
}.
\end{align}

Additionally, given the existence of auxiliary qubits $n_a$, we can further optimise and reduce the circuit depth for $n$-controlled NOT gates, given $n \geq 5$ and $n_a = n-2$. Further details of this decomposition are given in~\cite[Sec.~7.1]{barenco_elementary_1995}.

For completeness, we implement the above 3 decomposition strategies:
\begin{itemize}
\item Default quadratic decomposition as defined by circuit~\eqref{eqn:3cu_decomp} for $n \geq 2$ controls.
\item Improved 3CU decomposition for $n \eq 3$ controls.
\item Linear decomposition of $n \geq 5$ controls using $a\geq n-2$ auxiliary qubits.
\end{itemize}

In addition, due to ability to map CX gates to CZ and vice versa, we implement strategies to ensure all gate decompositions result in diagonal (controlled $\sqrt[k]{Z}$) gates. This ensures compatibility with currently existing simulator and circuit composition suites (e.g IBM Qiskit). As such, we can easily verify these methods and compare with results using existing platforms.

To aid with the decomposition of large NCU structures, the caching of intermediate gates is performed to minimise repeated calculations.
 
\subsection{Testing procedure}

\subsubsection{Validation}
Starting with a qubit register of $\vert 1 \rangle^{\otimes n}$, we can make use of an $n$ controlled unitary gate, where the control qubits are indexed as $c \in \{q_1,..,q_{n} \}$, and the target qubit as $ t = q_n+1$. We calculate the submatrix decomposition of the unitary matrix $U$ as $V=\sqrt{U}$ and $V^{\dagger}=\sqrt{U}^{\dagger}$. From here, we can recursively call the decomposition routine to further decompose gates with $n>1$ control lines. The base case can then rely on the available 2 qubit $CU$ routines to apply the appropriately decomposed gate matrix.

For validating this we can examine the states for the appropriate target to flip from $\vert 0\rangle$ to $\vert 1 \rangle$. By defining all potential starting states for a given qubit register and applying the NCU operation, we should see the case of $\vert 1 \rangle^{\otimes n} \vert 0 \rangle$ become $\vert 1 \rangle^{\otimes n} \vert 1 \rangle$. Larger NCU circuits will require the use of the optimised modules to ensure reasonable computational complexity. For the purposes of simplicity, we can examine the application of the $n$-controlled Pauli-x gate (NCX) structure to a qubit register of $n+1$ gates, starting in state $\vert \psi \rangle = \vert 1\rangle^{\otimes n} \vert 0 \rangle$. The application of the NCX gate to the register will result in the following state change:

\begin{equation}
 \textrm{NCX }\vert \psi \rangle = \vert 1\rangle^{\otimes n+1}.
\end{equation}

For comparison, all other register states but the above will result in no operation being applied to the target qubit in the register.

\section{Grover search use-case}
The Grover search algorithm

\subsection{Introduction}
General idea


\subsection{Quantum circuit description and construction}
Require methods and how it works
 
\subsection{Testing procedure}
Why we do this and how to test

\subsubsection{Validation}
Discuss implemented use-case 

\subsubsection{Benchmarking}
Circuit depth and performance testing 

\subsubsection{Tasks}
Sketch of the tasks envisaged to implement this use cases are as follows:

\begin{enumerate}
\item Script to generate QASM for QFT circuit with a given number of qubits
\item Script to generate QASM for ansatz circuit of given size and parameters (or to allow simulator start from arbitrary input state)
\item Script to run simulator and calculate output amplitudes for a set of configurations
\item Script to compare amplitudes and estimate the accuracy of the simulation
\item Ability to accurately record run time
\item Collate timings and accuracies, plot results and document conclusions
\end{enumerate}

\bibliographystyle{habbrv}
\bibliography{library}

\end{document}
This is never printed
